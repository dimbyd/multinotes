% demo_multinotes_exercises.tex

%\PassOptionsToPackage{noanswers}{multinotes}
%\PassOptionsToPackage{blankanswers}{multinotes}

%------------------------------------------------
\documentclass{article}
\usepackage[english,cleanlook]{isodate}
%------------------------------------------------

% paper
\usepackage[margin=30mm, a4paper]{geometry}
\setlength{\parindent}{0em}
\setlength{\parskip}{0.5em}

% multinotes
\usepackage{multinotes}

\textcolour{black}
\backgroundcolour{yellow!4}
\proofcolour{blue}
\answercolour{red}
\showcolour{purple}

\framedblankboxes
\framedproofs
\framedanswers
\setboxrule{0.5pt}

\setstretchfactor{1.2} 
\setimagestretchfactor{1}

\usepackage{amsthm}
\newtheorem{exercise}{\exercisename}
\newtheorem{quiz}{\quizname}

\loadbabel

% images
\usepackage{graphicx}
\usepackage{caption}
\usepackage{subfigure}
\graphicspath{{./figures/}}
\DeclareGraphicsExtensions{.png,.pdf,.jpeg,.jpg}

% hyperref
\usepackage[hidelinks]{hyperref}
\setcounter{tocdepth}{2}

% misc packages for testing
\usepackage{lipsum}
\usepackage{graphicx}
\usepackage{fancyvrb}

% document info
\title{Typesetting lecture notes with {\tt multinotes.sty}}
\author{D Evans}

%------------------------------------------------
\begin{document}
\maketitle
\tableofcontents

\bigskip\hrule\bigskip

\exercisename

\bigskip\hrule\bigskip

%----------------------------------------
\subsection*{The {\tt answer} environment}
%----------------------------------------

\begin{itemize}
\item {\tt blankanswers} replaces the content by blank space. 
\item {\tt noanswers} removes the answer completely.
\end{itemize}

Here is an answer environment (might be removed).
\begin{answer}
\lipsum[1]
\end{answer}

This is produced by the following code.
\begin{Verbatim}[frame=single]
\begin{answer}
\lipsum[1]
\end{answer}
\end{Verbatim}

Here is an answer environment with a custom title (might be removed)

\renewcommand{\answername}{\sl Answer}
\begin{answer}
\lipsum[1]
\end{answer}

This is produced by the following code.
\begin{Verbatim}[frame=single]
\renewcommand{\answername}{\sl Answer}
\begin{answer}
\lipsum[1]
\end{answer}
\end{Verbatim}

%----------------------------------------
\section{Questions, choices and checkboxes}
%------------------------------------------------

The package replicates the list environments found in {\tt exam.cls}, namely {\tt questions}, {\tt parts}, {\tt subparts}, {\tt choices} and {\tt checkboxes} and makes them responsive to {\tt blankanswers} and {\tt noanswers}. For example,

\begin{Verbatim}[frame=single]
\begin{exercise}\label{exe:demo}
\begin{questions}
\question What is $7\times 8$? \ans{54}
\question State Pythagoras' theorem. \ans{$a^2+b^2=c^2$}
\end{questions}
\end{exercise}
\end{Verbatim}

%Here is a list of questions inside a (theorem-like) exercise environment.
%\begin{exercise}
%\begin{questions}
%\item[] % for a linebreak
%\question A
%\begin{parts}
%\part AA
%\begin{subparts}
%\subpart  AAA
%\subpart AAB
%\end{subparts}
%\part AB
%\begin{subparts}
%\subpart  ABA
%\subpart ABB
%\end{subparts}
%\end{parts}
%\question B
%\begin{parts}
%\part BA
%\begin{subparts}
%\subpart  BAA
%\subpart BAB
%\end{subparts}
%\part BB
%\begin{subparts}
%\subpart  BBA
%\subpart BBB
%\end{subparts}
%\end{parts}
%\end{questions}
%\end{exercise}

% exercise
\bigskip
Here is a list of questions inside a (theorem-like) exercise environment.
\begin{exercise}

\begin{questions}
\item[] % for a linebreak
\question First question.
\begin{parts}
\part First part \ans{Answer to first part.}
\part Second part
\begin{subparts}
\subpart First subpart. \ans{Answer to first subpart.}
\subpart Second subpart. \ans{Answer to second subpart.}
\end{subparts}
\part Third part \ans{Answer to third part.}
\end{parts}
\question Second question \ans{Answer to second question.}
\end{questions}
\end{exercise}
%
%--------------------
\subsection{Multiple choice questions}

In what year did Columbus first cross the Atlantic?
\begin{choices}
\choice 1490 
\choice 1491 
\correctchoice 1492 
\choice 1493 
\end{choices}

This is produced by the following code.
\begin{Verbatim}[frame=single]
In what year did Columbus first cross the Atlantic?
\begin{choices}
\choice 1490 
\choice 1491 
\correctchoice   1492 
\choice 1493 
\end{choices}
\end{Verbatim}

\subsection{Multiple answer questions}

Which of the following were members of the Beatles?
\begin{checkboxes}
\correctchoice John
\correctchoice Paul
\correctchoice George
\choice Zippy
\end{checkboxes}

This is produced by the following code.
\begin{Verbatim}[frame=single]
Which of the following were members of the Beatles?
\begin{checkboxes}
\correctchoice John
\correctchoice Paul
\correctchoice George
\choice Zippy
\end{checkboxes}
\end{Verbatim}

\bigskip
\begin{quiz}
Here is a (theorem-like) quiz with a fill-the-blanks question, a true-or-false question, a multiple choice question, and a multiple answer question.

\begin{Verbatim}[frame=single]
Roses are \fillbox{red}, violets are \fillbox{blue}.
\end{Verbatim}

\begin{questions}
% FIL
\question
Fill in the blanks: {\sl
	Roses are \fillbox{red}, violets are \fillbox{blue}.
}


% TF question
\question\label{qu:rain}
The rain in Spain falls mainly on the plain.
\begin{choices}
\correctchoice True 		
\choice False		
\end{choices}

% MC question
\question\label{qu:america}
In what year did Columbus first cross the Atlantic?
\begin{choices}
\choice 1490 
\choice 1491 
\correctchoice 1492
\choice 1493 
\end{choices}

% MA question 
\question\label{qu:beatles}
Which of the following were members of the Beatles?
\begin{checkboxes}
\correctchoice John
\correctchoice Paul
\correctchoice George
\choice Bingo
\end{checkboxes}

\end{questions}
\end{quiz}

%------------------------------------------------
%\newpage
\section{Prime numbers quiz}
%------------------------------------------------

\begin{questions}

\question
What are the prime factors of 42?
\begin{choices}
\choice $3\times 14$
\correctchoice $2\times 3\times 7$
\choice $2\times 21$
\end{choices}

\question
What are the prime factors of 150?
\begin{choices}
\correctchoice $2\times 3\times 5\times 5$
\choice $3\times 3\times 5\times 5$
\choice $6\times 25$
\end{choices}

\question
What is the LCM of 13 and 3?
\begin{choices}
\choice $13$
\choice $3$
\correctchoice $39$
\end{choices}

\question
What is the HCF of 32 and 24?
\begin{choices}
\choice $2$
\correctchoice $8$
\choice $4$
\end{choices}

\question
What is the LCM of 24 and 36?
\begin{choices}
\choice $12$
\correctchoice $72$
\choice $30$
\end{choices}

\question
What is the HCF of 104 and 136?
\begin{choices}
\correctchoice $8$
\choice $12$
\choice $13$
\end{choices}

\question
What is the HCF of $3\times 13$ and $2^2\times 13$?
\begin{choices}
\choice $9$
\choice $17$
\correctchoice $13$
\end{choices}

\newpage % << remove
\question
What is the LCM of $2^2\times 3^2\times 5$ and $2^3\times 3\times 5$?
\begin{choices}
\correctchoice $360$
\choice $16\,120$
\choice $30$
\end{choices}

\question
What is the HCF of $2\times 5\times 13$ and $22\times 13$?
\begin{choices}
\choice $13$
\correctchoice $26$
\choice $52$
\end{choices}

\question
Find the LCM of $3^2\times 11\times 13$ and $3\times 11^2$.
\begin{choices}
\choice $1287$
\correctchoice $14\,157$
\choice $429$
\end{choices}

\end{questions}


%------------------------------------------------
\end{document}
%------------------------------------------------


