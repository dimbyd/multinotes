% binotes_lang_demo.tex 
% demo file for binotes language options

%\PassOptionsToPackage{cy}{binotes}
\PassOptionsToPackage{en}{binotes}
%\PassOptionsToPackage{en,cy}{binotes}
%\PassOptionsToPackage{cy,en}{binotes}

%------------------------------------------------
\documentclass{article}
%------------------------------------------------
\usepackage[utf8]{inputenc}
\usepackage[english,cleanlook]{isodate}

% layout
\usepackage[a4paper]{geometry}
\geometry{left=20mm,right=20mm}
\geometry{top=30mm,bottom=30mm}
%\geometry{landscape}
\setlength{\parindent}{0em}
\setlength{\parskip}{0.5em}

% binotes
\usepackage{binotes}
\columnsep=0.1\textwidth
\columnratio{0.5}
\textcolour{black}
\backgroundcolour{yellow!03}
\enstyle{\color{teal}\sl}
\cystyle{\color{purple}\sl}
\newcommand{\ency}[2]{\eng{#1}\cym{#2}}
\newcommand{\cyen}[2]{\cym{#1}\eng{#2}}

% for testing
\usepackage{lipsum}

% nice title
\renewcommand{\maketitle}[3]{
    \begin{center}
    \LARGE #1\\[1ex]
    \large #2 \\[1ex]
    \normalsize #3
    \end{center}
}

%------------------------------------------------
\begin{document}
%------------------------------------------------

\begin{twocol}

\cym{\maketitle{Teiposod dogfennau ddwyieithog drwy {\tt binotes.sty}}{D Evans}{\today}}
\eng{\maketitle{Typesetting bilingual documents with {\tt binotes.sty}}{D Evans}{\today}}

\sync
\cym{\begin{abstract}\lipsum[3]\end{abstract}}
\eng{\begin{abstract}\lipsum[4]\end{abstract}}

\sync
\cym{\section{Cyflwyniad}}
\eng{\section{Introduction}}

\cym{Dyma ddogfen brawf ar gyfer teiposod dogfennau ddwyieithog gyda  {\tt binotes.sty}.}
\eng{This is a test document for typesetting bilingual documents with {\tt binotes.sty}.}

%------------------------------
\sync
\eng{\section{Declarations}}
\cym{\section{Datganiadau}}

\cy Dyma frawddeg.\zz
\en Here is a sentence.
\cy Dyma frawddeg arall.
\en Here is another sentence.
\zz

%------------------------------
\sync
\eng{\section{Macros}}
\cym{\section{Macros}}

\cym{Dyma frawddeg.}
\eng{Here is a sentence.} 
\cym{Dyma frawddeg arall.}
\eng{Here is another sentence.} 


%------------------------------
\sync
\cym{\section{Amgylcheddau}}
\eng{\section{Environments}}

\begin{cymraeg}
Cynnwys amgylchedd Gymraeg
\end{cymraeg}

\begin{english}
The contents of an English environment
\end{english}

%------------------------------
\sync
\cym{\section{Rhestrau}}
\eng{\section{Lists}}

\cym{\subsection{Rhestr tu fewn paracol}}
\eng{\subsection{List inside paracol}}

\eng{Example~1}\cym{Enghraifft~1}
\begin{cymraeg}
\begin{enumerate}
\item Eitem Un 
\item Eitem Dau
\end{enumerate}
\end{cymraeg}
\begin{english}
\begin{enumerate}
\item Item One 
\item Item Two
\end{enumerate}
\end{english}

\sync
\eng{Example~2}\cym{Enghraifft~2}
\sync
\begin{itemize}
\cym{\item Eitem Un}
\eng{\item Item One}
\cym{\item Eitem Dau}
\eng{\item Item Two}
\end{itemize}

\sync
\cym{\subsection{Paracol tu fewn rhestr}}
\eng{\subsection{Paracol inside list}}

\eng{This is better for numbering and alignment.}
\cym{Mae hyn yn well o ran rhifo a chyfluniad.}

\end{twocol}

\twocol\eng{Example~1}\cym{Enghraifft~1}\endtwocol
\begin{enumerate}
\begin{twocol}
\cym{\item Eitem Un}
\eng{\item Item One}
\cym{\item Eitem Dau}
\eng{\item Item Two}
\end{twocol}
\end{enumerate}

\twocol\eng{Example~2}\cym{Enghraifft~2}\endtwocol
\begin{enumerate}
\begin{twocol}
\cym{\item Eitem Un \item Eitem Dau}
\eng{\item Item One \item Item Two}
\end{twocol}
\end{enumerate}

\twocol\eng{Example~3}\cym{Enghraifft~3}\endtwocol
\begin{itemize}
\begin{twocol}
\cym{\item Eitem Un \item Eitem Dau}
\eng{\item Item One \item Item Two}
\end{twocol}
\end{itemize}

%------------------------------
\begin{twocol}
\eng{\section{Tables}}
\cym{\section{Tablau}}
\end{twocol}

Here is a single table with bilingual entries (fine for mono).\par
\begin{tabular}{|l|l|}
\hline
\cym{Dŵr}\eng{Water} 	& \cym{Blodau}\eng{Flowers} \\
\hline
\ency{Trees}{Coed}		& \cyen{Afon}{River} \\
\hline
\end{tabular}

Here are two monolingual tables, one on each side of a {\tt twocol} environment (needed for parallel)

\begin{twocol}
\cym{
\begin{tabular}{|l|l|}\hline
Dŵr & Blodau \\ \hline
Coed & Afon \\ \hline
\end{tabular}
}
\eng{
\begin{tabular}{|l|l|}\hline
Water & Flowers \\ \hline
Trees & River \\ \hline
\end{tabular}
}
\end{twocol}


%--------------------
\begin{twocol}
\eng{\section{Footnotes}}
\cym{\section{Troednodiadau}}

\en Footnotes can be moody!
\cy Gall troednodiadau beri problemau!
\end{twocol}
\begin{twocol}
\eng{A footnote here\footnote{Hello world!}.}
\cym{Troednodyn fan hyn\footnote{Helo byd!}.}

\end{twocol}

%------------------------------
\appendix

%------------------------------
\begin{twocol}
\eng{\section{Paragraphs}}
\cym{\section{Paragraffau}}
\cym{\lipsum[1]}
\eng{\lipsum[2]}
\end{twocol}

%------------------------------------------------
\end{document}
%------------------------------------------------


