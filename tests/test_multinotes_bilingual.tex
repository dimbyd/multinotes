% multinotes_test.tex
% Test file for multinotes package

\documentclass{article}
\usepackage[T1]{fontenc}
\usepackage[english,cleanlook]{isodate}

% layout
\usepackage{geometry}

\setlength{\parindent}{0em}
\setlength{\parskip}{0.3em}
\setlength\parsep{\parskip}

% basic packages
\usepackage{color}
\usepackage{lipsum}

% multinotes 
\usepackage{multinotes}
\definelanguage{welsh}{cy}{cym}{cymraeg}
\definelanguage{breton}{br}{bre}{brezhoneg}
\definelanguage{english}{en}{eng}{english}
\loadbabel

\setlanguagestyle{welsh}{\color{red}}
\setlanguagestyle{breton}{\color{purple}}
\setlanguagestyle{english}{\color{blue}\slshape}
\setcommonstyle{\normalfont\color{black}}

% monolingual
%\geometry{a4paper,portrait,margin=25mm}
%\includelanguage{welsh}
%\includelanguage{english}

% bilingual
\geometry{a3paper,landscape,margin=30mm}
\includelanguage{welsh}
%\includelanguage{breton}
\includelanguage{english}

% multicols options
\multicolson
\columnsep=0.1\textwidth
\columnseprule=0.4pt
\raggedright
\localcounter{section}
\localcounter{enumi}

% troubleshoot spaces
%\renewcommand{\normalfont}{\ttfamily}

% define bilingual commands
% marked as the first and second languages in order of inclusion.
% TODO: finish and move to multinotes.sty
\newcommand{\bititle}[2]{\cym{#1}\eng{#2}}

% meta information
\title{\cym{Croeso}\eng{Welcome}}
\author{DE}
\date{\cym{\today}\eng{\today}}

%\input{runtime-options}

%--------------------
\begin{document}
\maketitle

\makeatletter
\@ifundefined{c@multinotes@welshnumber}{Welsh not included}{Welsh included}\\
\@ifundefined{c@multinotes@bretonnumber}{Breton not included}{Breton included}\\
\@ifundefined{c@multinotes@englishnumber}{English not included}{English included}
\makeatother

\bigskip\hrule\bigskip


% abstracts 
\begin{multicols}

\begin{cymraeg}
\begin{abstract}
\lipsum[1]
\end{abstract}
\end{cymraeg}

\begin{english}
\begin{abstract}
\lipsum[2]
\end{abstract}
\end{english}

\end{multicols}

\bigskip\hrule\bigskip

\section{Sequential mode: declarations, macros and environments}

1. \cy Helo \en Hello \zz

\bigskip\hrule\bigskip

2. {\cy Helo} {\en Hello} 

\bigskip\hrule\bigskip

3. {\cy Helo \en Hello}

\bigskip\hrule\bigskip

4. \cym{Helo} \eng{Hello}

\bigskip\hrule\bigskip

5.
\begin{cymraeg}
Helo
\end{cymraeg}
\begin{english}
Hello
\end{english}

\bigskip\hrule\bigskip

%------------------------------
% test multicols 
\section{{\tt multicols} and  {\tt\textbackslash mcspan}}

%\multicolson

\begin{multicols}

\cy cy-dec \en en-dec \zz

\cym{cym-mac} \eng{eng-mac} 

\begin{cymraeg}
cymraeg-env
\end{cymraeg}

\begin{english}
englsh-env
\end{english}

\mcspan{\bigskip\lipsum[5]\bigskip} %

\cy cy-dec2 \en en-dec2 \zz

\cym{cym-mac2} \eng{eng-mac2} 

\begin{cymraeg}
cymraeg-env2
\end{cymraeg}

\begin{english}
english-env2
\end{english}

\end{multicols}


%------------------------------
% test mcsync
\section{{\tt\textbackslash mcsync}}

\begin{multicols}

\begin{cymraeg}
\lipsum[1]
\end{cymraeg}

\begin{english}
\lipsum[2]
\end{english}

\mcsync %

\cym{\lipsum[5]}
\eng{\lipsum[6]}

\mcspan{\bigskip\lipsum[11]\bigskip} %

\begin{cymraeg}
\section{Cymraeg}
\lipsum[5]
\end{cymraeg}

\begin{english}
\section{English}
\lipsum[6]
\end{english}

\mcsync % 

\cym{\lipsum[7]}
\eng{\lipsum[8]}

\cym{\lipsum[9]}
\eng{\lipsum[10]}

\end{multicols}

%------------------------------
% multicols on/off
\section{{\tt\textbackslash multicolson} and {\tt\textbackslash multicolsoff}}

\multicolsoff
\begin{multicols}
\cym{\lipsum[11]}
\eng{\lipsum[12]}
\end{multicols}
\multicolson

%------------------------------
% numbering
\section{Numbering}

\begin{multicols}
%
\begin{cymraeg}
\subsection{Rhestri rhifedig}
\begin{enumerate}
\item Cyntaf
\item Ail
\item Trydydd
\end{enumerate}
\end{cymraeg}
%
\begin{english}
\subsection{Numbered lists}
\begin{enumerate}
\item First
\item Second
\item Third
\end{enumerate}
\end{english}
%
\end{multicols}

\end{document}

